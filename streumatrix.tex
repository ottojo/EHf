\chapter{Streumatrix}

\begin{equation}
    \vb = \MS \cdot \va
\end{equation}

Die Streumatrix besteht aus Streuparametern und beschreibt den Zusammenhang der $N$ Ausgänge $\vb$ mit
den $N$ Eingängen $\va$ eines $N$-Tors.

\section{Leistungswellen}
Hier wird die Überlagerung einer hin- und rücklaufenden Spannungs- bzw. Stromwelle durch Normierung mit $\sqrt{Z_0}$ mit den Parametern $\underline{a}$ und $\underline{b}$ dargestellt. Die Einheit der Parameter ist $\sqrt{\text{Leistung}}$.
\begin{equation}
    \uu = \frac{1}{\sqrt{Z_0}} \left( \UU_h + \UU_r \right) = \frac{\UUh}{\sqrt{Z_0}} e^{-\ugamma z} + \frac{\UUr}{\sqrt{Z_0}} e^{\ugamma z} = \uu_h + \uu_r = \underline{a} + \underline{b}
\end{equation}
\begin{equation}
    \ii = \II \sqrt{Z_0} =  \ii_h - \ii_r = \uu_h + \uu_r = \underline{a} - \underline{b}
\end{equation}
\begin{align}
    \underline{a} &= \frac{1}{2} (\uu + \ii)\\
    \underline{b} &= \frac{1}{2} (\uu - \ii)
\end{align}

\subsection{Wirkleistungstransport}
\todo{Poynting Vektor?}
Bei monofrequenter Anregung gilt
\begin{equation}
    P_w = \frac{1}{2} \left( \abs{a}^2 - \abs{b}^2 \right)
\end{equation}

\section{Eigenschaften von N-Toren}
\begin{enumerate}
    \item Reziprozität:\\
        Ein N-Tor ist reziprok, wenn Streumatrix symmetrisch: $\us_{ij} = \us_{ji} \;  \forall i\neq j$
    \item Symmetrie:\\
        Ein reziprokes N-Tor ist symmetrisch, wenn gilt $\us_{ii} = \us_{jj} \; \forall i,j$
    \item Verlustlos:\\
        Ist das N-Tor verlustlos, gilt $\MS \cdot \MS ^{*T} = \overleftrightarrow{1}$
\end{enumerate}
