\chapter{Rauschen}

\section{Ursachen}
\begin{description}
    \item[Thermisches Rauschen] tritt auf, da die Bewegung freier Ladungsträger durch
        Gitterschwingungen in ihrer Geschwindigkeit fluktuiert.
    \item[Schrotrauschen] wird durch die Quantisierung von Ladungsträgern verursacht. Auch ist die
        Generation und Rekombination von Ladungsträgern bei Halbleitern nicht über die Zeit konstant.
    \item[$1/f$-Rauschen] tritt bei der Emmision von Elektronen aus einer Oberfläche auf, ist aber
        bei Frequenzen oberhalb \SI{10}{\kilo\hertz} gegenüber den anderen Rauschquellen zu
        vernachlässigen.
    \item[Funkelrauschen] wird oft mit dem $1/f$-Rauschen gleichgesetzt.
\end{description}
Im Folgenden wird nur thermisches Rauschen angenommen.

\section{Thermisches Rauschen}
Die \emph{Rauschleistungsdichte $w$} berechnet sich näherungsweise als
\begin{equation}
    \boxed{w \approx kT}
\end{equation}
mit der \emph{Bolzmann-Konstante $k=k_B$} und der \emph{Temperatur $T$}.
Unter der Annahme, dass die Rauschleistungsdichte $w$ im Frequenzintervall $B$ konstant ist, gilt
für die \emph{Rauschleistung $N$}
\begin{equation}
    \boxed{N = wB \approx kTB}
\end{equation}

\subsection{Ersatzrauschquelle}

\subsection{Rauschzahl}
Die Leistung $P$ eines empfangenen Signals besteht aus Signalleistung $S$ und Rauschleistung $N$.
Damit lässt sich auch der SNR Wert definieren:
\begin{gather}
    P=S+N\\
    \SNR = \frac{S}{N}
\end{gather}
Die \emph{Rauschzahl $F$} wird nun als Verhältnis der Rauschzahlen am Ein- und Ausgang eines Zweitors
definiert:
\begin{equation}
    \boxed{F = \frac{\SNR_\ein}{\SNR_\aus}}
\end{equation}
In \si{\decibel} umgerechnet ergibt sich das \emph{Rauschmaß} \NF:
\begin{equation}
    \NF = 10\log_{10} F \; \si{\decibel}
\end{equation}
Hat das Zweitor einen frequenzunabhängigen \emph{Gewinn $G$}, so nimmt man an dass dieser das
Eingangssignal verstärkt und danach zusätzliche Rauschleistung $N_\text{zus}$ addiert wird:
\begin{equation}
    P_\aus = GP_\ein + N_\text{zus} = \underbrace{GS_\ein}_{S_\aus} + \underbrace{GN_\ein + N_\text{zus}}_{N_\aus}
\end{equation}
Für die Rauschzahl in Abhängigkeit von Rauschleistung $N$ und Gewinn $G$ ergibt sich
\begin{equation}
    F = 1 + \frac{N_\text{zus}}{N_\ein G}
\end{equation}
Da die Rauschzahl von der Eingangsleistung abhängt, wird diese normiert auf die
\emph{ Standardrauschzahl $F_0$}, bei Temperatur $T_0 = \SI{290}{\kelvin}$ bzw Eingangsrauschleistung
$N_\ein = kT_0B$:
\begin{equation}
    F = 1 + \underbrace{\frac{N_\text{zus}}{kT_0B \cdot G}}_{F_\text{zus}}
\end{equation}
Im Folgenden ist mit $F$ die Standardrauschzahl $F_0$ gemeint.

Die zusätzliche Rauschleistung am Ausgang $N_\text{zus}$ lässt sich auch als Rauschleistung am Eingang
$N_i$ darstellen:
\begin{equation}
    N_i = \frac{N_\text{zus}}{G}
\end{equation}
Damit wird die äquivalente Rauschtemperatur $T_e$ bestimmt:
\begin{equation}
    N_i = kT_eB
\end{equation}
Die Rauschzahl abhängig von $N_i$ lautet
\begin{equation}
    F = 1 + \frac{N_i}{kT_0B}
\end{equation}
Durch Einsetzen der letzten beiden Gleichungen folgt
\begin{equation}
    F = 1 + \frac{T_e}{T_0}
\end{equation}








